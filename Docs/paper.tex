%\documentclass[9pt,twocolumn]{IEEEtran}
%\documentclass[12pt,twoside,draft]{IEEEtran}
%\documentclass[12pt,draft]{IEEEtran}
%\documentclass[11pt,draft]{IEEEtran}
%\documentclass[11pt,final]{IEEEtran}
%\documentclass[11pt,ds,final]{my_IEEEtran}
%\documentclass[11pt,onecolumn,draftcls]{IEEEtran}
%\documentclass[11pt]{IEEEtran}
\documentclass[11pt,onecolumn]{IEEEtran}
%\documentclass[draft]{IEEEtran}
%\documentclass[9pt,twocolumn,technote,twoside]{IEEEtran}

\def\BibTeX{{\rm B\kern-.05em{\sc i\kern-.025em b}\kern-.08em
    T\kern-.1667em\lower.7ex\hbox{E}\kern-.125emX}}

%\topmargin 0.125in

\usepackage{times}
\usepackage{subfigure}
%\usepackage{algorithmic}
\usepackage{algpseudocode}
\usepackage{algorithm}
%\renewcommand{\algorithmicrequire}{\textbf{Input:}}
%\renewcommand{\algorithmicensure}{\textbf{Output:}}

%\usepackage{epsfig}
%\usepackage{graphics,graphicx,fancyhdr,amsfonts,amsmath,color,epic}
\usepackage[dvips]{graphics,graphicx}
%\usepackage{graphicx,graphics}
\DeclareGraphicsExtensions{.jpg,.pdf,.eps,.png}

\newtheorem {lemma}{Lemma}
\newtheorem {theorem}{Theorem}

\begin{document}

\title{An OpenCL Implementation of Wait-Free Sets}

\author{Sudarshan Shinde \\
        Bangalore, INDIA. \\
Email:sudarshan\_shinde@iitbombay.org}

\maketitle

%\markboth{IEEE Transactions On Automatic Control, Vol. XX, No. Y, Month 1999}
%{Murray and Balemi: Using the style file IEEEtran.sty}

%\noindent EDICS No :DSP-BANK, DSP-MULT, DSP-TFSR.

%\thispagestyle{plain}
%\pagestyle{plain}

\begin{abstract}

\end{abstract}

\begin{IEEEkeywords}
wait-free programming, multithreading, gpgpu, opencl.
\end{IEEEkeywords}

\footnotetext{Copyright (c) 2010 IEEE. Personal use of this material is permitted. However, permission to use this material for any other purposes must be obtained from the IEEE by sending a request to pubs-permissions@ieee.org.}

\pagebreak

\section{Implementation Details}

The algorithms described in the literarure do not describe memory management. In perticular, they do not take into consideration, what happens if a node is physically deleted, while another process it using that node.

OpenCL 2.0 defines Shared Virtual Memory (SVM), that could be used to share data between host and the GPU device, without explicit data transfer. However OpenCL does not support dynamic data allocation on GPU side, and requires that size of SVM be pre-defined. This restricts maximum size of a set.

Since an implementation needs to take care of memory management, in this implementation we allocate a node pool of size $N$ in SVM. Though this restricts set size to be less than $N$, it still highlights many implementation aspects of wait-free set implementation in heterogenous computing setting.  

\section{The Algorithm}
The algorithm consists of a modification of the wait-free linked list to support replacing a node.

A {\it packed} reference consists of an unmarked reference, a {\it mark} bit to indicate if the node is marked and an {\it retain} bit to indicate that unmarked reference could not be changed. It also has a {\it free} bit to indicate that whether the node is occupied or free. These three bits are arranged as $Bits = [fBit,rBit,mBit]$.

We also have the following methods

\begin{enumerate}
    \item $[ref, bits] = unpackRef(pref)$.
\end{enumerate} 

\begin{algorithm}
  \caption{Snips next node if it is marked and could be snipped}
  \label{alg:snip}
  
  \begin{algorithmic}[1]
    %\Require
    %    \Statex startRef:Reference to the starting node.
    %\Ensure
    %   \Statex  next node will be snipped.
    %\Statex   
    \Function{snip}{startRef}
      \State pPRef = startRef:next;
      \State [nRef,pBits] = unpackRef(startRef:next);
      \If{(pBits = [0,0,x])}
        \State [nnRef,nBits] = unpackRef(nRef:next):
        \If{(nBits = [0,x,1])}
          \State status = CAS(startRef:next,pPRef,[nnRef,pBits]);
          \If{(status = true)}
             \State FREE(nRef);
             \State 
             \Return {\it snipped};
          \Else
             \State
             \Return {\it failed};
          \EndIf
        \Else
          \State
          \Return {\it next\_unmarked};
        \EndIf
      \Else
        \State
        \Return {\it start\_invalid};
      \EndIf
    \EndFunction
  \end{algorithmic}
\end{algorithm}

\begin{algorithm}
  \caption{Replaces next node with a new node}
  \label{alg:replace}
  
  \begin{algorithmic}[1]
    %\Require
    %    \Statex startRef:Reference to the starting node.
    %\Ensure
    %   \Statex next node will be replaced by new node.
    \Statex   
    \Function{replace}{startRef, newRef}
      \State pPRef = startRef:next;    
      \State [nRef,pBits] = unpackRef(startRef:next);
      \If{([pBits] = [0,0,0])}
        \State nPRef = nRef:next;    
        \State [nnRef,nBits] = unpackRef(nRef:next);
        \If{(nBits = [0,x,0])}
          \State status = CAS(nRef:next,nPRef,[nnRef,[0,1,0]]);
          \If{(status = false)}
            \State
            \Return {\it retain\_failed};
          \EndIf  
        \Else
          \State
          \Return {\it invalid\_nbits};
        \EndIf
        \State newRef:next = [nnRef,[0,0,0]];
        \State status = CAS(startRef:next,pPRef,[newRef,pBits]);
        \If{(status = true)}
          \State FREE(nRef);
          \State
          \Return {\it replaced}; 
        \Else
          \State
          \Return {\it failed};
        \EndIf
      \Else
        \State
        \Return {\it invalid\_pbits};
      \EndIf
    \EndFunction
  \end{algorithmic}
\end{algorithm}

\begin{algorithm}
  \caption{cleans all the nodes that are logically deleted and could be physically deleted}
  \label{alg:clean}
  \begin{algorithmic}[1]
    %\Require
    %    \Statex startRef:Reference to the starting node.
    %\Ensure
    %   \Statex cleans all the logically deleted nodes.
    \Statex   
    \Function{clean}{startRef, nextRef\&}
      \State pRef  = startRef;
      \State nextRef  = {\it null};
      \While{(true)}
        \State status = snip(pRef);
        \State [nRef,pBits] = unpackRef(pRef:next);
        \If{(status = {\it next\_unmarked})}
          \State nextRef = nRef;
          \State
          \Return status;
        \ElsIf{status = {\it start\_invalid}}
          \If{(pBits = [0,1,x])}
            \State pRef = nRef; 
          \Else
            \If{(pRef = startRef)}
              \State
              \Return {\it start\_invalid}
            \Else
              \State pRef = startRef;  
            \EndIf
          \EndIf
        \EndIf
      \EndWhile
    \EndFunction
  \end{algorithmic}
\end{algorithm}

\begin{algorithm}
  \caption{checks for valid window.}
  \label{alg:window}
  \begin{algorithmic}[1]
    %\Require
    %    \Statex startRef:Reference to the starting node.
    %\Ensure
    %   \Statex cleans all the logically deleted nodes.
    \Statex   
    \Function{window}{key, prevRef, nextRef, index\&}
      \State status = false;
      \If{(prevRef != {\it Null})}
        \State [nRef, pBits] = unpackRef(prevRef:next);
        \If{(pBits != [0,x,0])}
          \State 
          \Return {\it invalid\_prev\_bits}; ; 
        \EndIf
        \If{(nRef != nextRef)}
          \State 
          \Return {\it invalid\_next\_ref}; 
        \EndIf
        \State pMaxVal = simdMAX(prevRef:val);  
        \If{(key $<=$ pMaxVal)}
          \State 
          \Return {\it window\_not\_found};; 
        \EndIf
      \EndIf
      \State [nnRef, nBits] = unpackRef(nextRef:next);
      \If{(nBits != [0,x,0])}
        \State 
        \Return {\it invalid\_next\_bits}; 
      \EndIf
      \State pMaxVal = simdMAX(nextRef:val);  
      \If{(key $<=$ pMaxVal)}
        \State 
        \Return {\it window\_not\_found};
      \Else
        \If{(simdANY((key = nextRef:Val), index))}
          \State
          \Return {\it key\_found}
        \Else
          \State
          \Return {\it window\_found}
        \EndIf
      \EndIf
    \EndFunction
  \end{algorithmic}
\end{algorithm}

\begin{algorithm}
  \caption{finds a key and returns a window to it.}
  \label{alg:find}
  \begin{algorithmic}[1]
    %\Require
    %    \Statex startRef:Reference to the starting node.
    %\Ensure
    %   \Statex cleans all the logically deleted nodes.
    \Statex   
    \Function{find}{key, prevRef\&, nextRef\&, index\&}
      \State startRef = hash(key);
      \State prevRef = nextRef = {\it Null};
      \State index = 0;      
      \While{(true)}
        \State pRef = {\it Null};
        \State nRef = startRef;
        \State status = {\it window\_not\_found};
        \While{(status = {\it window\_not\_found})}
          \State status = WINDOW(key, pRef, nRef, index);
          \If{(status = {\it window\_not\_found})}
            \State pRef = nRef;
            \State status2 = CLEAN(pRef,nRef);
            \If{(status2 = {\it start\_invalid})}
              \State status = {\it window\_not\_found}
            \EndIf
          \EndIf
        \EndWhile
        \If{(status = {\it window\_found})}
          \State
          \Return status;
        \EndIf
        \If{(status = {\it key\_found})}
          \State
          \Return status;
        \EndIf
      \EndWhile
    \EndFunction
  \end{algorithmic}
\end{algorithm}

\begin{algorithm}
  \caption{clones a node and adds a key to it.}
  \label{alg:clone}
  \begin{algorithmic}[1]
    %\Require
    %    \Statex startRef:Reference to the starting node.
    %\Ensure
    %   \Statex cleans all the logically deleted nodes.
    \Statex   
    \Function{clone}{ref, key}
      \State newRef = ALLOC();
      \Comment{assumption is that ALLOC always returns a node}
      \State newRef:next = ref:next;
      \State keyIndex = simdIndex(ref:val = key);
      \If{keyIndex = 0}
        \State minIndex    = min(simdIndex(ref:val = {\it empty}));
        \Comment{simdIndex returns zero if condition is not met}         
        \If{(minIndex $\not =$ 0)}
          \State simdCopy(ref:val, newRef:val);
          \State newRef:val[minIndex] = key;
          \State
          \Return newRef;
        \Else
          \State prevNewRef = ALLOC();
          \State simdCopy(ref:val, prevNewRef:val, (ref:val < key));
          \State simdCopy(ref:val, newRef:val, (ref:val > key));
          \State prevNewRef:next = [newRef,[0,0,0]];
          \State minIndex    = min(simdIndex(newRef:val = {\it empty}));      
          \State newRef:val[minIndex] = key;
          \State
          \Return prevNewRef;
        \EndIf
      \Else
        \State simdCopy(ref:val, newRef:val);
        \State newRef:val[keyIndex] = {\it empty};
        \State
        \Return newRef;
      \EndIf  
      \State
      \Return {\it null};
    \EndFunction
  \end{algorithmic}
\end{algorithm}

\begin{algorithm}
  \caption{traverses the list from prev ref to next ref.}
  \label{alg:traverse}
  \begin{algorithmic}[1]
    %\Require
    %    \Statex startRef:Reference to the starting node.
    %\Ensure
    %   \Statex cleans all the logically deleted nodes.
    \Statex   
    \Function{traverse}{prevRef, nextRef}
      \State pRef = prevRef;
      \State [nRef, pBits] = unpackRef(pRef:next);
      \While{(pBits = [0,x,x])}
        \If{(nRef $\not = $ nextRef)}
          \State pRef = nRef;
          \State [nRef, pBits] = unpackRef(pRef:next);
        \Else
          \State
          \Return pRef;
        \EndIf
      \EndWhile
      \State
      \Return {\it null};
    \EndFunction
  \end{algorithmic}
\end{algorithm}

\begin{algorithm}
  \caption{adds a key to the set.}
  \label{alg:add}
  \begin{algorithmic}[1]
    %\Require
    %    \Statex startRef:Reference to the starting node.
    %\Ensure
    %   \Statex cleans all the logically deleted nodes.
    \Statex   
    \Function{add}{key}
      \While{(true)}
        \State status = FIND(key, prevRef nextRef,index);
        \If{status = {\it window\_found}}
          \State pRef = TRAVERSE(prevRef, nextRef);
          \If{(pRef $\not =$ {\it null})}
            \State cloneRef = CLONE(nextRef,key);
            \State status = REPLACE(pRef,cloneRef);
            \If{(status = {\it replaced})}
              \State
              \Return {\it true};
            \EndIf
          \EndIf
        \Else
          \State
          \Return {\it false}
        \EndIf
      \EndWhile
    \EndFunction
  \end{algorithmic}
\end{algorithm}

\begin{algorithm}
  \caption{removes a key from the set.}
  \label{alg:remove}
  \begin{algorithmic}[1]
    %\Require
    %    \Statex startRef:Reference to the starting node.
    %\Ensure
    %   \Statex cleans all the logically deleted nodes.
    \Statex   
    \Function{remove}{key}
      \While{(true)}
        \State status = FIND(key, prevRef nextRef,index);
        \If{status = {\it key\_found}}
          \State pRef = TRAVERSE(prevRef, nextRef);
          \If{(pRef $\not =$ {\it null})}
            \State cloneRef = CLONE(nextRef,key);
            \State status = REPLACE(pRef,cloneRef);
            \If{(status = {\it replaced})}
              \State
              \Return {\it true};
            \EndIf
          \EndIf
        \Else
          \State
          \Return {\it false}
        \EndIf
      \EndWhile
    \EndFunction
  \end{algorithmic}
\end{algorithm}


%\section{Proof of Correctness}


%\Bibliographystyle{ieeetr}
%\include{references.bbl}


%%%%%\bibliography{bib-file}  % commented if *.bbl file included, as seen below

\end{document}


%%%%%%%%%%%%%%%%%%%%%%%%%%%  End of IEEEsample.tex  %%%%%%%%%%%%%%%%%%%%%%%%%%%
